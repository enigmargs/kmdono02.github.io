%%%%%%%%%%%%%%%%%%%%%%%%%%%%%%%%%%%%%%%%%
% Medium Length Professional CV
% LaTeX Template
% Version 2.0 (8/5/13)
%
% This template has been downloaded from:
% http://www.LaTeXTemplates.com
%
% Original author:
% Rishi Shah 
%
% Important note:
% This template requires the resume.cls file to be in the same directory as the
% .tex file. The resume.cls file provides the resume style used for structuring the
% document.
%
%%%%%%%%%%%%%%%%%%%%%%%%%%%%%%%%%%%%%%%%%

%----------------------------------------------------------------------------------------
%	PACKAGES AND OTHER DOCUMENT CONFIGURATIONS
%----------------------------------------------------------------------------------------

\documentclass{resume} % Use the custom resume.cls style

\usepackage{longtable}
\usepackage{bibentry}
\usepackage[maxbibnames=99, style=nature]{biblatex}
\usepackage{hyperref}

\usepackage[left=0.75in,top=0.6in,right=0.75in,bottom=0.6in]{geometry} % Document margins
\newcommand{\tab}[1]{\hspace{.2667\textwidth}\rlap{#1}}
\newcommand{\itab}[1]{\hspace{0em}\rlap{#1}}
\name{Kevin Donovan} % Your name
\address{119C Cole St., Chapel Hill, NC 27516} % Your address
\address{(315)727-3603 \\ kmdono02@ad.unc.edu} % Your phone number and email

\addbibresource{CV.bib}
\addbibresource{CV_software.bib}
\addbibresource{CV_talks.bib}

\newcommand*{\boldname}[3]{%
  \def\lastname{#1}%
  \def\firstname{#2}%
  \def\firstinit{#3}}
\boldname{}{}{}

% Patch new definitions
\renewcommand{\mkbibnamegiven}[1]{%
  \ifboolexpr{ ( test {\ifdefequal{\firstname}{\namepartgiven}} or test {\ifdefequal{\firstinit}{\namepartgiven}} ) and test {\ifdefequal{\lastname}{\namepartfamily}} }
  {\mkbibbold{#1}}{#1}%
}

\renewcommand{\mkbibnamefamily}[1]{%
  \ifboolexpr{ ( test {\ifdefequal{\firstname}{\namepartgiven}} or test {\ifdefequal{\firstinit}{\namepartgiven}} ) and test {\ifdefequal{\lastname}{\namepartfamily}} }
  {\mkbibbold{#1}}{#1}%
}

\boldname{Donovan}{Kevin}{K.}

\DeclareFieldFormat[misc]{title}{#1}
\DeclareFieldFormat[proceedings]{title}{#1}

\begin{document}

%----------------------------------------------------------------------------------------
%	EDUCATION SECTION
%----------------------------------------------------------------------------------------

\begin{rSection}{Education}

{\bf University of North Carolina at Chapel Hill} \hfill {\em August 2015 - Present} 
\\ PhD in Biostatistics
\\ Department of Biostatistics
\\ Gillings School of Global Public Health\\
\\{\bf Syracuse University} \hfill {\em January 2013 - May 2015} 
\\ B.S. in Mathematics\hfill { GPA: 3.962}
\\ B.S. with Distinction in Economics

\end{rSection}

%----------------------------------------------------------------------------------------
%	Objective
%----------------------------------------------------------------------------------------

\begin{rSection}{Objective}
Leading statistical analyses and teaching statistics in a collaborative setting, along with the development of methods for analyzing spatio-temporal data.  I am interested in analyzing associations between spatial locations and how these may change across time for geographical and brain imaging data.
\end{rSection}

%----------------------------------------------------------------------------------------
%	Research Interests
%----------------------------------------------------------------------------------------

\begin{rSection}{Research Interests}
Spatial Data Analysis\\
Time Series Analysis\\
Neural Imaging Data Analysis\\
Network Analysis\\
Statistical Signal Processing\\
Causal Inference\\
Machine Learning
\end{rSection}

%--------------------------------------------------------------------------------
%    Experience
%-----------------------------------------------------------------------------------------------
\begin{rSection}{Experience}
{\bf Research Assistant} \hfill {\em March 2018 - Present}
\\ Carolina Institute for Developmental Disabilities
\begin{itemize}
    \item Development of algorithms for early prediction of Autism Spectrum Disorder (ASD) using behavioral data and imaging data, with random forests, support vector machines, and deep learning methods using R and Python.  Random forest algorithm using behavioral data published.
    \item Analysis focused on examining causes of ASD prevalence and symptom heterogeneity by infant sex, using latent variable models such as factor analysis and growth mixture models.
    \item Development of a set of tutorials detailing the use of R software for data management and data analysis.  Course based on these tutorials created with bi-weekly virtual sessions held and corresponding office hours.
    \item Direct collaboration with scientists writing statistical analysis and results sections in published manuscripts.  Further duties included data management using R, writing code in R for all corresponding statistical analysis, and creation of figures and tables using R.  Methods used include generalized linear models, mixed models with longitudinal data, mediation models, and non-supervised clustering algorithms.
\end{itemize}
{\bf Teaching Assistant} \hfill {\em August 2017 - December 2017} 
\\ BIOS 600: Principles of Statistical Inference
\begin{itemize}
    \item Teaching assistant for introductory statistics class for non-Biostatistics public health graduate students
    \item Organized and ran lab sessions with 50+ students.  Sessions consisted of practice applying statistical principals to real and simulated data using R computing software.
    \item Graded lab reports, held office hours and review sessions for mid term and final examinations
\end{itemize}
{\bf Research Assistant} \hfill {\em September 2016 - May 2019} 
\\ Collaborative Studies Coordinating Center (CSCC)
\begin{itemize}
    \item Under direction of mentor, lead statistical analyses for published research on HIV-positive youth, directly collaborating with investigators across the United States.  Responsibilities included data management using SAS and R, writing code in R for all statistical analyses, creation of figures and tables using R, and communicating the results and methods to investigators.
    \item Development of R package \textbf{lodr} containing software to conduct regression analyses when some predictors have a known limit of detection, requiring the use of Rcpp and C++ code.  Package made publicly available on CRAN.
\end{itemize}
{\bf Research Assistant} \hfill {\em August 2015 - March 2018} 
\\ Dr. Michael G. Hudgens
\begin{itemize}
    \item Developed and published research on methodology for estimating biomarker levels which correspond to a desired upper bound on the risk of disease, with corresponding R code for implementing the methods published on Github.
\end{itemize}

\end{rSection}
%----------------------------------------------------------------------------------------
%	Coursework
%----------------------------------------------------------------------------------------

\begin{rSection}{Coursework}
Advanced Probability and Statistical Inference\\
Linear and Generalized Linear Models\\
Longitudinal Data Analysis\\
Statistical Methods in Diagnostic Medicine\\
Machine Learning\\
Survival Analysis\\
Spatial Statistics
\end{rSection}

%----------------------------------------------------------------------------------------
%	Computing Experience
%----------------------------------------------------------------------------------------

\begin{rSection}{Computing Experience}
R, SAS, C++ and Rcpp, Matlab, Linux cluster computing

\end{rSection}

%----------------------------------------------------------------------------------------
%	Developed Software
%----------------------------------------------------------------------------------------

\begin{rSection}{Developed Software}

\begin{refsection}[CV_software.bib]
\nocite{*}
\leavevmode\printbibliography[omitnumbers=true,heading=none]
\end{refsection}

\end{rSection}

%----------------------------------------------------------------------------------------
%	Publications
%----------------------------------------------------------------------------------------
\begin{rSection}{Publications}

\begin{refsection}[CV.bib]
\nocite{*}
\printbibliography[keyword=Published,omitnumbers=true,title=Published]
\printbibliography[keyword=Accepted,omitnumbers=true,title=Accepted]
\printbibliography[keyword=Submitted,omitnumbers=true,title=Submitted]
\end{refsection}

\end{rSection}

%----------------------------------------------------------------------------------------
%	Professional Presentations
%----------------------------------------------------------------------------------------
\begin{rSection}{Professional Presentations}

\begin{refsection}[CV_talks.bib]
\nocite{*}
\leavevmode\printbibliography[omitnumbers=true,heading=none]
\end{refsection}

\end{rSection}

%----------------------------------------------------------------------------------------
%	References
%----------------------------------------------------------------------------------------
\begin{rSection}{References}
Available upon request
\end{rSection}

\end{document}
